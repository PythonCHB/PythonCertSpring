\documentclass{beamer}
%\usepackage[latin1]{inputenc}
\usetheme{Warsaw}
\title[Python Certificate: System Development]{Week 6: Advanced OO -- Numpy}
\author{Christopher Barker}
\institute{UW Continuing Education}
\date{April  30, 2013}

\usepackage{listings}
\usepackage{hyperref}

\begin{document}

% ---------------------------------------------
\begin{frame}
  \titlepage
\end{frame}

% ---------------------------------------------
\begin{frame}
\frametitle{Table of Contents}
%\tableofcontents[currentsection]
  \tableofcontents
\end{frame}


\begin{frame}[fragile]{Some updates:}

\Large{Some folks have asked to learn about Desktop GUIs}

\vfill
\Large{And the class is smaller than it may have been}

\vfill
\Large{So: week 9 will be Desktop GUIS with wxPython}

\vfill
\Large{And we'll do all the student presentations on week 10: May 28th}


\end{frame}

\begin{frame}[fragile]{BlueBox VMs}

\Large{Does anyone still need their BlueBox VMs?}


\vfill
\large{ ( Conor and Tyler, you're all set) }

\end{frame}

\section{Advanced-OO}

% ---------------------------------------------
\begin{frame}[fragile]{class creation}

{\Large What happens when a class instance is created?}

\vfill
\begin{verbatim}
class Class(object):
    def __init__(self, arg1, arg2):
        self.arg1 = arg1
        self.arg2 = arg2
        .....
\end{verbatim}
\vfill

\begin{itemize}
  \item A new instance is created
  \item \verb|__init__| is called
  \item The code in \verb|__init__| is run to initialize the instance
\end{itemize}
\vfill

\end{frame} 

% ---------------------------------------------
\begin{frame}[fragile]{class creation}

{\Large What if you need to do something before creation?}

\vfill
{\Large Enter: \verb|__new__|}

\vfill
\begin{verbatim}
class Class(object):
    def __new__(cls, arg1, arg2):
        some_code_here
        return cls()
        .....
\end{verbatim}

\vfill
\begin{itemize}
  \item \verb|__new__| is called: it returns a new instance
  \item The code in \verb|__new__| is run to pre-initialize
  \item \verb|__init__| is called
  \item The code in \verb|__init__| is run to initialize the instance
\end{itemize}
\vfill

\end{frame} 


% ---------------------------------------------
\begin{frame}[fragile]{class creation}

{\large \verb|__new__| is a static method -- but it must be called with a class object as the first argument. And it should return a class instance: }

\vfill
\begin{verbatim}
class Class(superclass):
    def __new__(cls, arg1, arg2):
        some_code_here
        return superclass.__new__(cls)
        .....
\end{verbatim}

\vfill
\begin{itemize}
  \item \verb|__new__| is called: it returns a new instance
  \item The code in \verb|__new__| is run to pre-initialize
  \item \verb|__init__| is called
  \item The code in \verb|__init__| is run to initialize the instance
\end{itemize}
\vfill

\end{frame} 

% ---------------------------------------------
\begin{frame}[fragile]{class creation}

{\Large When would  you need to use it:}

\begin{itemize}
  \item subclassing an immutable type:\\
   -- It's too late to change it once you get to \verb|__init__|

  \item When \verb|__init__| not called:
  \begin{itemize}
    \item unpickling
    \item copying
  \end{itemize}

  {\large You may need to put some code in \verb|__new__| to make sure things go right} 

\end{itemize}

\vfill
{\large More detail here:}
\url{http://www.python.org/download/releases/2.2/descrintro/#__new__}
\end{frame} 

% ---------------------------------------------
\begin{frame}[fragile]{LAB}

{\large Demo: \verb|code/__new__/new_example.py|}

\vfill
{\Large Write a subclass of int that will always be an even number: round the input to the closest even number}

\vfill
{\large \verb|code/__new__/even_int.py|}

\vfill
\end{frame}


% ---------------------------------------------
\begin{frame}[fragile]{multiple inheritance}

{\Large Multiple inheritance:\\
\hspace{0.2in} Pulling from more than one class}

\vfill
\begin{verbatim}
class Combined(Super1, Super2, Super3):
    def __init__(self, something, something else):
        Super1.__init__(self, ......)        
        Super2.__init__(self, ......)        
        Super3.__init__(self, ......)        
\end{verbatim}
(calls to the super classes \verb|__init__| are optional -- case dependent)

\end{frame} 

% ---------------------------------------------
\begin{frame}[fragile]{multiple inheritance}

\vfill
{\Large Method Resolution Order -- left to right}

\begin{enumerate}
  \item Is it an instance attribute ?
  \item Is it a class attribute ?
  \item Is it a superclass attribute ?
  \begin{enumerate}
     \item is it an attribute of the left-most superclass?
     \item is it an attribute of the next superclass?
     \item ....
  \end{enumerate}
  \item Is it a super-superclass attribute ?
  \item ...also left to right...
\end{enumerate}

\vfill
{\large ( This can get complicated...more on that later...)}

\end{frame} 

% ---------------------------------------------
\begin{frame}[fragile]{mix-ins}

{\Large Why would you want to do this?}

\vfill
{\Large Hierarchies are not always simple:}
\vfill
\begin{itemize}
  \item Animal
  \begin{itemize}
    \item Mammal
    \begin{itemize}
      \item GiveBirth()
    \end{itemize}
    \item Bird
    \begin{itemize}
      \item LayEggs()
    \end{itemize}
  \end{itemize}
\end{itemize}
\vfill
{\Large Where do you put a Platypus or an Armadillo?}

\vfill
{\Large Real World Example: \verb|wxPython FloatCanvas|}
\end{frame} 


%--------------------------------
\begin{frame}[fragile]{super}

{\Large getting the superclass:}
\begin{verbatim}
class SafeVehicle(Vehicle):
    """
    Safe Vehicle subclass of Vehicle base class...
    """
    def __init__(self, position=0, velocity=0, icon='S'):
        Vehicle.__init__(self, position, velocity, icon)
\end{verbatim}

{\Large
\vfill
not DRY
\vfill
also, what if we had a bunch of references to superclass?
}
\end{frame} 

\begin{frame}[fragile]{super}

{\Large getting the superclass:}
\begin{verbatim}
class SafeVehicle(Vehicle):
    """
    Safe Vehicle subclass of Vehicle base class
    """
    def __init__(self, position=0, velocity=0, icon='S'):
        super(SafeVehicle, self).__init__(position, velocity, icon)
\end{verbatim}

\vfill
{\Large but super is about more than just DRY...}

\vfill
{\Large Remember the method resolution order?}
\end{frame} 


\begin{frame}[fragile]{What does super() do?}

\vfill
{\Large \verb|super| returns a \\
``proxy object that delegates method calls''}

\vfill
{\Large It's not returning the object itself -- but you can call methods on it}

\vfill
{\Large It runs through the method resolution order (MRO) to find the method you call.}

\vfill
{\Large Key point: the MRO is determined \emph{at run time} }

\vfill
\url{http://docs.python.org/2/library/functions.html#super}

\end{frame} 

\begin{frame}[fragile]{What does super() do?}

\vfill
{\Large Not the same as calling one superclass method...}

\vfill
{\Large \verb|super()| will call all the sibling superclass methods}

\begin{verbatim}
class D(C, B, A):
    def __init__(self):
       super(D, self).__init__()
\end{verbatim}
same as
\begin{verbatim}
class D(C, B, A):
    def __init__(self):
       C.__init__()
       B.__init__()
       A.__init__()
\end{verbatim}

{\Large you may not want that...}

\end{frame} 



\begin{frame}[fragile]{super}

{\Large Two seminal articles about \verb|super()|:}

\vfill
{\LARGE ``Super Considered Harmful''}\\[0.1in]
{\Large \hspace{0.5in}-- James Knight }

\vfill
\url{https://fuhm.net/super-harmful/}


\vfill
{\LARGE ``super() considered super!''}\\[0.1in]
{\Large  \hspace{0.5in}--  Raymond Hettinger }

\vfill
\url{http://rhettinger.wordpress.com/2011/05/26/super-considered-super/}
\vfill

{\large (Both worth reading....)}
\end{frame} 

\begin{frame}[fragile]{super issues...}

{\Large Both actually say similar things:}

\begin{itemize}
  \item The method being called by super() needs to exist
  \item Every occurrence of the method needs to use super():\\
        Use it consistently, and document that you use it, as it is part of the external interface for your class, like it or not.
  \item The caller and callee need to have a matching argument signature:
  \begin{itemize}
     \item Never call super with anything but the exact arguments you received, unless you really know what you're doing.
     \item When you use it on methods whose acceptable arguments can be altered on a subclass via addition of more optional arguments, always accept *args, **kwargs, and call super like \verb|super(MyClass, self).method(args_declared, *args, **kwargs)|.
  \end{itemize}

\end{itemize}

\end{frame} 

%\section{meta-classes}

%-------------------------------
\begin{frame}[fragile]{Wrap Up}

{\LARGE Thinking OO in Python:}

\vfill
{\large Think about what makes sense for your code:}
\begin{itemize}
  \item {\large Code re-use}
  \item {\large Clean APIs}
  \item {\large ... }
\end{itemize}

\vfill
{\large Don't be a slave to what OO is \emph{supposed} to look like. }

\vfill
{\large Let OO work for you, not \emph{create} work for you}

\end{frame}


%-------------------------------
\begin{frame}[fragile]{Wrap Up}

{\Large OO in Python:}

\vfill
{\Large The Art of Subclassing}: Raymond Hettinger

\vfill
{\small \url{http://pyvideo.org/video/879/the-art-of-subclassing}}

\vfill
''classes are for code re-use -- not creating taxonomies''

\vfill
{\Large Stop Writing Classes}: Jack Diederich

\vfill
{\small \url{http://pyvideo.org/video/880/stop-writing-classes}}

\vfill
``If your class has only two methods -- and one of them is \verb|__init__|
-- you don't need a class ''
\end{frame}


\section{numpy}

%---------------------------
\begin{frame}[fragile]{numpy}

\vfill
\begin{center}
{\LARGE numpy}

\vfill
{\Large Not just for lots of numbers!\\[0.1in]
(but it's great for that!)}
\end{center}
\vfill
\url{http://www.numpy.org/}
\end{frame} 


\begin{frame}[fragile]{what is numpy?}

{\Large An N-Dimensional array object}

\vfill
{\Large A whole pile of tools for operations on/with that object.}

\vfill

\end{frame} 

%----------------------------------
\begin{frame}[fragile]{Why numpy?}

{\Large Classic answer: Lots of numbers}

\vfill
\begin{itemize}
  \item Faster
  \item Less memory
  \item More data types
\end{itemize}

\vfill
{\Large Even if you don't have lot of numbers:}
\begin{itemize}
  \item N-d array slicing
  \item Vector operations
  \item Flexible data types
\end{itemize}

\end{frame} 

% %----------------------------------
\begin{frame}[fragile]{why numpy?}

{\Large Wrapper for a block of memory:}

\begin{itemize}
  \item Interfacing with C libs
  \item PyOpenGL
  \item GDAL
  \item NetCDF4
  \item Shapely
\end{itemize}

{\Large Image processing:}
\begin{itemize}
  \item PIL
  \item WxImage
  \item ndimage
\end{itemize}

\end{frame} 

%----------------------------------
\begin{frame}[fragile]{Whatis an nd array?}
\begin{itemize}
  \item N-dimensional (up to 32!)
  \item Homogeneous array:
  \begin{itemize}
    \item Every element is the same type\\
          (but that type can be a pyObject)
    \item Int, float, char -- more exotic types
  \end{itemize}
  \item ``rank'' – number of dimensions
  \item Strided data:
  \begin{itemize}
    \item Describes how to index into block of memory
    \item PEP 3118 -- Revising the buffer protocol
  \end{itemize}
\end{itemize}

\vfill
{\large demo: \verb|structure.py|}

\end{frame} 

%----------------------------------
\begin{frame}[fragile]{Built-in Data Types}

\begin{itemize}
  \item Signed and unsigned Integers\\
        8, 16, 32, 64 bits
  \item Floating Point\\
        32, 64, 96, 128 bits (not all platforms)
  \item Complex\\
        64, 128, 192, 256 bits
  \item String and unicode\\
        Static length 
  \item Bool \\
        8 bit
  \item Python Object \\
        Really a pointer
\end{itemize}

\end{frame} 

%----------------------------------
\begin{frame}[fragile]{Compund dtypes}

{\Large
\begin{itemize}
  \item Can define any combination of other types \\
        Still Homogenous:  Array of structs.
  \item Can name the fields
  \item Can be like a database table
  \item Useful for reading binary data
\end{itemize}
}

\vfill
{\Large demo: \verb|dtypes.py|}

\end{frame} 

%----------------------------------
\begin{frame}[fragile]{Array Constructors:}

{\Large From scratch:}

\verb|ones(), zeros(), empty(), arange(), linspace(), logspace()|

( Default dtype: np.float64 )


\vfill
{\Large From sequences:}

\verb|array(), asarray()|

( Build from any sequence )

\vfill
{\Large  From binary data:}
\verb|fromstring(), frombuffer(), fromfile()|

\vfill
{\Large Assorted linear algebra standards:}
\verb|eye(), diag(), etc.| 

\vfill
{\large demo: \verb|constructors.py|}
\end{frame} 

%----------------------------------
\begin{frame}[fragile]{Broadcasting:}

{\Large Element-wise operations among two different rank arrays:}

\vfill
{\Large Simple case: scalar and array:}

\begin{verbatim}
In [37]: a
Out[37]: array([1, 2, 3])

In [38]: a*3
Out[38]: array([3, 6, 9])
\end{verbatim}

{\Large Great for functions of more than one variable on a grid}

\vfill
{\large demo: \verb|broadcasting.py|}
\end{frame} 

%----------------------------------
\begin{frame}[fragile]{Working with compiled code:}

{\Large Wrapper around a C pointer to a block of data}

\begin{itemize}
  \item Some code can't be vectorized
  \item Interface with existing libraries
\end{itemize}

\vfill
{\Large Tools:}
\begin{itemize}
  \item C API: you don't want to do that!
  \item Cython: typed arrays
  \item Ctypes 
  \item SWIG: numpy.i 
  \item Boost: boost array
  \item f2py
\end{itemize}

\vfill
Example of numpy+cython: \url{http://wiki.cython.org/examples/mandelbrot}
\end{frame} 

%----------------------------------
\begin{frame}[fragile]{numpy persistance:}

{\Large \verb|.tofile() / fromfile()|}\\
 -- Just the raw bytes, no metadata

\vfill
{\Large pickle }

\vfill
{\Large \verb|savez()| -- numpy zip format}\\
Compact: binary dump plus metadata

\vfill
{\Large netcdf}

\vfill
{\Large Hdf}
\begin{itemize}
  \item Pyhdf
  \item pytables
\end{itemize}


\end{frame} 

%----------------------------------
\begin{frame}[fragile]{Other stuff:}

\begin{itemize}
  \item Masked arrays
  \item Memory-mapped files
  \item Set operations: unique, etc
  \item Random numbers
  \item Polynomials
  \item FFT
  \item Sorting and searching
  \item Linear Algebra
  \item Statistics
\end{itemize}

{\large (And all of scipy!)}

\end{frame} 

%----------------------------------
\begin{frame}[fragile]{numpy docs:}

\url{www.numpy.org}\\
   -- Numpy reference Downloads, etc

\vfill
\url{www.scipy.org}\\
   -- lots of docs

\vfill
{\large Scipy cookbook}\\
\url{http://www.scipy.org/Cookbook}

\vfill
{\large ``The Numpy Book''}\\
\url{http://www.tramy.us/numpybook.pdf}


\end{frame} 

% %----------------------------------
% \begin{frame}[fragile]{Other stuff:}

% \end{frame} 



%-------------------------------
\begin{frame}[fragile]{Next Week:}

\vfill
{\LARGE Threading / Multiprocessing}\\[0.2in]
{\Large  \hspace{0.5in} -- Jeff}


\vfill
{\Large And of course, your projects...}

\vfill

\end{frame}


\end{document}
 